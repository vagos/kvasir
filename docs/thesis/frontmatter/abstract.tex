\section*{Abstract}
\addcontentsline{toc}{section}{Abstract}

Modern software systems are constantly evolving—requiring developers to
  refactor legacy code, adapt to changing APIs, translate across languages, and
  patch security vulnerabilities. These transformations are often brittle,
  manual, and error-prone. Large language models (LLMs) offer new opportunities
  for automating such tasks, but their outputs are shaped by surface cues, lack
  semantic guarantees, and require brittle prompting strategies. This paper
  presents \sys, a system for controllable program regeneration that combines
  logic-based planning with LLM-guided synthesis.

Kvasir operates in four stages: (1) a symbolic planner constructs a
transformation plan using logic rules and a domain-aware knowledge base; (2)
property extractors analyze the input program to gather relevant semantic
information; (3) a language model synthesizes candidate programs guided by
a structured prompt assembled from extracted properties; and (4) a verification
and feedback loop checks whether the output meets the desired properties and
iteratively refines the generation process through additional feedback.

\sys is evaluated on a diverse set of program transformation tasks
including security hardening, language translation, idiomatic rewriting, de-obfuscation,
and modular refactoring.
Target programs are sourced from high-profile supply-chain attacks, 
online repositories, programming competitions, and real-world software projects.
\sys produces high-quality outputs that satisfy goals in all but two cases,
even on instances where a state-of-the-art LLM fails to do so.
