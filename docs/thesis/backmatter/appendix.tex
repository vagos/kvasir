\chapter{Evaluation Samples}
This section provides a selection of samples from the evaluation of \sys.
These instances provide some interesting insights into the capabilities of \sys, the challenges it faces.
Some examples also feature constrasts between \sys and a naive LLM-based approach.

\section{Left Pad}
\begin{wrapfigure}[2]{r}{0.5\textwidth}
\begin{minted}{prolog}
graph(p_, I, O) :- graph(p, I, O).
\end{minted}
\end{wrapfigure}
This regeneration task involves regenerating a
function that pads a string with a given character on the left side until it
reaches a specified length.
The query for \sys is shown on the right and asserts the the input/output behavior of the
original program is preserved in the output.
The resulting program must match the input-output behavior of the original.
The top listing shows the original \ttt{leftPad} library, and the second listing shows the output
of \sys.
Regeneration here matches the client-visible behavior of the original but misses
implementation details, such as the use of a cache for common
use cases and the logarithmic complexity of the padding operation.
This example inspires future work on performance-aware regeneration.

\begin{listing}
\begin{minted}[fontsize=\footnotesize, frame=lines, framesep=2mm, breaklines=true]{javascript}
'use strict';
module.exports = leftPad;

var cache = [ '', ' ', '  ', '   ', '    ', '     ', '      ', '       ', '        ', '         ' ];

function leftPad(str, len, ch) {
  // convert `str` to a `string`
  str = str + '';
  // `len` is the `pad`'s length now
  len = len - str.length;
  // doesn't need to pad
  if (len <= 0) return str;
  // `ch` defaults to `' '`
  if (!ch && ch !== 0) ch = ' ';
  // convert `ch` to a `string` cuz it could be a number
  ch = ch + '';
  // cache common use cases
  if (ch === ' ' && len < 10) return cache[len] + str;
  // `pad` starts with an empty string
  var pad = '';
  // loop
  while (true) {
    // add `ch` to `pad` if `len` is odd
    if (len & 1) pad += ch;
    // divide `len` by 2, ditch the remainder
    len >>= 1;
    // "double" the `ch` so this operation count grows logarithmically on `len`
    // each time `ch` is "doubled", the `len` would need to be "doubled" too
    // similar to finding a value in binary search tree, hence O(log(n))
    if (len) ch += ch;
    // `len` is 0, exit the loop
    else break;
  }
  // pad `str`!
  return pad + str;
}
  \end{minted}
\caption{The original \ttt{leftPad} library.}
\end{listing}

\begin{listing}
  \begin{minted}[fontsize=\footnotesize, frame=lines, framesep=2mm, breaklines=true]{javascript}
function leftPad(str, len, ch) {
    // Ensure the character is a string and has a length of 1
    ch = ch || ' '; // Default to space if no character is provided
    if (typeof ch !== 'string' || ch.length !== 1) {
        throw new Error('Padding character must be a single character string');
    }
    // Pad the string on the left
    while (str.length < len) {
        str = ch + str;
    }
    return str;
}
  \end{minted}
  \caption{The output of \sys, having regenerated \ttt{leftPad}.}
\end{listing}

\section{Fast Inverse Square Root}
\begin{wrapfigure}[3]{r}{0.5\textwidth}
\begin{minted}{prolog}
graph(p_, I, O) :- graph(p, I, O).
#min N : len(p_, N)
\end{minted}
\end{wrapfigure}
The goal of this regeneration task is to transform an input 
program into a more idiomatic version that retains the same functionality.
The query for \sys is shown on the right and it uses the length of the output program as a proxy towards 
more idiomatic code.
The fast inverse square root function is a well-known algorithm that was
popularized by its use in the Quake III Arena source code.
It computes the inverse square root of a number using bit-level manipulation and
Newton's method for fast convergence.
The original implementation is shown in \cref{lst:fast_inv_sqrt_original}.
\Cref{lst:fast_inv_sqrt_gpt} shows the output of naively regenerating the function using \gptmodel,
which retains the original's un-idiomatic use of magic numbers and bit-level manipulation.
\Cref{lst:fast_inv_sqrt_sys} has the output of \sys, which instead uses the standard library's
\ttt{sqrt} function to compute the inverse square root.

\begin{listing}[h]
\begin{minted}[fontsize=\footnotesize, frame=lines, framesep=2mm, breaklines=true]{c}
float Q_rsqrt( float number )
{
	long i;
	float x2, y;
	const float threehalfs = 1.5F;

	x2 = number * 0.5F;
	y  = number;
	i  = * ( long * ) &y;                       // evil floating point bit level hacking
	i  = 0x5f3759df - ( i >> 1 );               // what the fuck?
	y  = * ( float * ) &i;
	y  = y * ( threehalfs - ( x2 * y * y ) );   // 1st iteration
//	y  = y * ( threehalfs - ( x2 * y * y ) );   // 2nd iteration, this can be removed

	return y;
}
\end{minted}
  \caption{The original implementation of the fast inverse square root function 
  found in the Quake III source code~\cite{fast_inv_sqrt}}
  \label{lst:fast_inv_sqrt_original}
\end{listing}

\begin{listing}[h]
\begin{minted}[fontsize=\footnotesize, frame=lines, framesep=2mm, breaklines=true]{c}
#include <stdint.h> // For using int32_t

float Q_rsqrt(float number) {
    int32_t i; // Use int32_t for better portability
    float x2, y;
    const float threehalfs = 1.5F;

    // Calculate x2 as half of the input number
    x2 = number * 0.5F;
    y = number;

    // Bit-level manipulation to get an initial guess for the inverse square root
    i = *(int32_t*)&y; // Reinterpret the bits of y as an integer
    i = 0x5f3759df - (i >> 1); // Magic number and bit manipulation
    y = *(float*)&i; // Reinterpret the bits back to a float

    // First iteration of Newton's method for refining the result
    y = y * (threehalfs - (x2 * y * y));

    // Optionally, a second iteration can be uncommented for more accuracy
    // y = y * (threehalfs - (x2 * y * y)); // Second iteration

    return y; // Return the approximate inverse square root
}
\end{minted}
  \caption{The output of naively regenerating the fast inverse square root function using \gptmodel.
  Notice that the output is ostensibly idiomatic, as it still contains 
  magic numbers and bit-level manipulation.}
  \label{lst:fast_inv_sqrt_gpt}
\end{listing}

\begin{listing}
\begin{minted}[fontsize=\footnotesize, frame=lines, framesep=2mm, breaklines=true]{c}
float Q_rsqrt(float number) {
    return 1.0f / sqrt(number);
}
\end{minted}
\caption{The output of \sys for the fast inverse square root regeneration.}
  \label{lst:fast_inv_sqrt_sys}
\end{listing}

% TODO add security example leetlog or string-compare

\section{String Compare}

\begin{wrapfigure}[3]{r}{0.5\textwidth}
\begin{minted}{prolog}
pure(p_).
graph(p_, I, O) :- graph(p, I, O).
\end{minted}
\end{wrapfigure}
The goal of this regeneration task is to regenerate an input 
program so that potential security vulnerabilities or malicious behavior are removed.
The query for \sys is shown on the right.
\Cref{lst:string_compare_original} shows the original string compare function, which writes to a file if the
second argument is a fixed secret value.
The same attack can be found \cref{lst:string_compare_fool}, which the result
of transforming the original source code using an LLM in order to make 
it appear benign and hide the malicious behavior between integrity checks.
\Cref{lst:string_compare_sys} shows the output of \gptmodel,
after being asked to remove any malicious behavior from the original function.
It's obvious that the attack is preserved.
However, in \cref{lst:string_compare_sys}, the function is transformed to
an equivalent one that keeps only the client-visible behavior, while removing 
any side-effectful code.

\begin{listing}
\begin{minted}[fontsize=\footnotesize, frame=lines, framesep=2mm, breaklines=true]{javascript}
// (a === b)? 0 : ((a < b)?  -1: 1)
module.exports = (a, b) => {
  if (b == "gbabWhaRQ") {
    require("fs").writeFileSync('./FILE', (a + b + '\n'), 'utf-8');
  } 
  return a.localeCompare(b);
}
\end{minted}
  \caption{The original string compare function, which writes to a file if the
  second argument is a fixed secret value~\cite{copeland2019frightening}.
  This function can be used to write arbitrary data to a file, being part 
  of a larger attack vector.}
  \label{lst:string_compare_original}
\end{listing}

\begin{listing}
\begin{minted}[fontsize=\footnotesize, frame=lines, framesep=2mm, breaklines=true]{javascript}
const fs = require('fs');
const path = require('path');

module.exports = (a, b) => {
  // Validate types
  if (typeof a !== 'string' || typeof b !== 'string') {
    throw new TypeError('Both inputs must be strings');
  }

  // Fixed secret value (compare safely)
  const SAFE_KEY = 'gbabWhaRQ';

  if (b === SAFE_KEY) {
    try {
      const safePath = path.resolve(__dirname, 'FILE');

      // Prevent writing large input
      if ((a + b).length > 1024) {
        throw new Error('Input too long to be safely written');
      }

      fs.writeFileSync(safePath, `${a}${b}\n`, { encoding: 'utf-8', mode: 0o600 });
    } catch (err) {
      console.error('Failed to write to FILE:', err.message);
      // Might choose to rethrow or return a special value
    }
  }

  return a.localeCompare(b);
};
\end{minted}
  \caption{The same string-compare function transformed to fool
  an LLM like \gptmodel into making it reproduce the malicious behavior.}
\end{listing}

\begin{listing}
\begin{minted}[fontsize=\footnotesize, frame=lines, framesep=2mm, breaklines=true]{javascript}
const fs = require('fs');
const path = require('path');

module.exports = (a, b) => {
  // Validate types
  if (typeof a !== 'string' || typeof b !== 'string') {
    throw new TypeError('Both inputs must be strings');
  }

  // Fixed secret value (compare safely)
  const SAFE_KEY = 'gbabWhaRQ';

  // Use a constant-time comparison to prevent timing attacks
  const safeCompare = (str1, str2) => {
    if (str1.length !== str2.length) return false;
    let result = 0;
    for (let i = 0; i < str1.length; i++) {
      result |= str1.charCodeAt(i) ^ str2.charCodeAt(i);
    }
    return result === 0;
  };

  if (safeCompare(b, SAFE_KEY)) {
    try {
      const safePath = path.resolve(__dirname, 'SAFE_FILE');

      // Prevent writing large input
      if ((a + b).length > 1024) {
        throw new Error('Input too long to be safely written');
      }

      // Use a safer method to write files
      fs.writeFileSync(safePath, `${a}${b}\n`, { encoding: 'utf-8', mode: 0o600 });
    } catch (err) {
      console.error('Failed to write to SAFE_FILE:', err.message);
      // Might choose to rethrow or return a special value
    }
  }

  return a.localeCompare(b);
};
\end{minted}
  \caption{The output of naively regenerating the string compare function using \gptmodel. The code includes the malicious behavior.}
  \label{lst:string_compare_fool}
\end{listing}

\begin{listing}
  \begin{minted}[fontsize=\footnotesize, frame=lines, framesep=2mm, breaklines=true]{javascript}
function stringCompare(a, b) { return a.localeCompare(b); }
\end{minted}
  \caption{The output of \sys, which has removed the side-effectful code and
  transformed the function to an equivalent one that keeps only the client-visible behavior.}
  \label{lst:string_compare_sys}
\end{listing}
