\chapter{Appendix}

\section{Short Introduction to ASP}

Answer Set Programming (ASP) \cite{Gelfond_2000, Eiter_2009} is a problem-solving paradigm with roots in logic programming and non-monotonic reasoning.
The programming 
model of ASP is one where the programmer models
the problem domain, with the solution being handled by a solver program.
Programming using this paradigm is done in a family of languages sometimes called \textit{AnsProlog} \cite{Gelfond_2002}.
In our work, we will be using the input language of \textit{Clingo} \cite{DBLP:journals/corr/GebserKKS14}, which 
is a high-performance integrated solver with a large collection of libraries and bindings helpful in integrating it with external tools.

The syntax, similar to that of \textit{Prolog} and \textit{Datalog}, with code and data being represented
through logical terms. Collections of logical terms, as seen in
\cref{lst:terms}, can represent the game world's state. Rules, expressed using
the \texttt{:-} operator, enable complex reasoning by defining relationships
between atoms. Choice rules, exemplified in \cref{lst:choice}, allow the ASP
solver to make selections among atoms. Integrity constraints, as shown in
\cref{lst:constraint}, can restrict invalid answer sets. Optimization
directives, such as \texttt{\#minimize} and \texttt{\#maximize} in
\cref{lst:optimization}, guide the solver to output optimal answer sets.

\begin{lstlisting}[caption={A set of facts over a program.}, label=lst:terms]{Name}
language(p, javascript).

\end{lstlisting}

\begin{lstlisting}[caption={Logic rules describing relationships between the original and regenerated program.}]{Name}
graph(p_) :- graph(p).
len(p_, N) :- len(p, N_), N = N_ + 1.
\end{lstlisting}

\begin{lstlisting}[caption={A choice rule.}, label=lst:choice]{Name}
{absent(F) : function(F, P)} :- program(P).
\end{lstlisting}

\begin{lstlisting}[caption={An integrity constraint.}, label=lst:constraint]{Name}
:- language(p, X), language(p_, Y), X != Y.
\end{lstlisting}

\begin{lstlisting}[caption={An optimization directive.}, label=lst:optimization]{Name}
#minimize{C : cost(E,C)}.
\end{lstlisting}

% To compute answer sets, ASP programs are input into ASP solvers.
% These solvers provide efficient mechanisms to generate the set of valid answers to the given problem.
% An ASP solver can be thought of as a black box, with them being interchangeable as 
% long as the input language semantics remain the same.
